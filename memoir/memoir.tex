\documentclass[ebook, 10pt, openright, onecolumn]{memoir}
\usepackage[T1]{fontenc}
\usepackage{graphicx}
\usepackage[utf8]{inputenc}
\usepackage[english]{babel}
\usepackage[final]{microtype}
\usepackage{todonotes}
\title{On the Verge of Poor}
\author{Tyler Thomas}

\setlrmarginsandblock{0.15\paperwidth}{*}{1} % Left and right margin
\setulmarginsandblock{0.2\paperwidth}{*}{1}  % Upper and lower margin
\checkandfixthelayout

\def\signed #1{{\leavevmode\unskip\nobreak\hfil\penalty50\hskip2em
  \hbox{}\nobreak\hfil(#1)%
  \parfillskip=0pt \finalhyphendemerits=0 \endgraf}}

\newsavebox\mybox
\newenvironment{aquote}[1]
  {\savebox\mybox{#1}\begin{quote}}
  {\signed{\usebox\mybox}\end{quote}}


\newlength{\drop}
%% specify the Webomints family
\newcommand*{\wb}[2]{\fontsize{#1}{#2}\usefont{U}{webo}{xl}{n}}
%% select a (FontSite) font by its font family ID
\newcommand*{\FSfont}[1]{\fontencoding{T1}\fontfamily{#1}\selectfont}
%% if you don’t have the FontSite fonts either \renewcommand*{\FSfont}[1]{}
%% or use your own choice of family.
%% select a (TeX Font) font by its font family ID
\newcommand*{\TXfont}[1]{\fontencoding{T1}\fontfamily{#1}\selectfont}
%% Generic publisher’s logo
\newcommand*{\plogo}{\fbox{$\mathcal{T}$}}

\newcommand*\td[1]{
  \todo[inline]{
     #1 
  }
}

%% Define some commands for marking up the chapters 
\newcommand*\starbreak{\fancybreak*{\Large{* * *}}}
\newcommand*\finish{\td{ ----- Finish this section -----}}



\newcommand*{\titleAT}{\begingroup  % Anatomy of a Typeface
  \thispagestyle{empty}
  \FSfont{5bp} % FontSite Bergamo (Bembo)
  \drop=0.1\textheight
  \vspace*{\drop}
  \rule{\textwidth}{1pt}\par
  \vspace{2pt}\vspace{-\baselineskip}
  \rule{\textwidth}{0.4pt}\par
  \vspace{0.5\drop}
  \centering
  {\FSfont{5ml} % FontSite Mona Lisa
    \Huge ON THE VERGE OF POOR}\\[0.5\baselineskip]
  {\FSfont{5ml}
    \Large a memoir}\par
  \vspace{0.25\drop}
  \rule{0.3\textwidth}{0.4pt}\par
  \vspace{\drop}
  {\Large \scshape Tyler Thomas}\par
  \vfill
  {\large\scshape $T^2$ Press}\par
  \vspace*{\drop}
  \endgroup
}


%% At some point kill the chapter numbering in this pagestyle
\addtopsmarks{headings}{}{
  \createmark{chapter}{both}{shownumber}{}{. \ }
}
\pagestyle{headings}

\let\cleardoublepage\clearpage

%%%%%%%%%%%%%%%%%%%%%%%%%%%%%%%%%%%%%%%%%%%%%%%%%%%%%%%%%%%%%
%%%                     BEGIN DOCUMENT                    %%%
%%%%%%%%%%%%%%%%%%%%%%%%%%%%%%%%%%%%%%%%%%%%%%%%%%%%%%%%%%%%%
\begin{document}

\pagestyle{headings}
\titleAT
\clearpage

\frontmatter

\null\vfill

\begin{flushleft}
  \textit{On the Verge of Poor}

  © COPYRIGHT Tyler Thomas

  2019
  
  ISBN--INFO

  ISBN--13: 
  \bigskip

  ALL RIGHTS RESERVED  
\end{flushleft}

\clearpage
\tableofcontents
\clearpage

\mainmatter

\chapter{Desert}
\label{cha:desert}
\begin{aquote}{Latex Memoir Class Documentation}
  \textbf{memoir}, \textit{n.} a fiction designed to flatter the subject and to
  impress the reader.
\end{aquote}


I remember the desert most.  Days of walking down roads which led nowhere in the
hot sand and silence, my form of nihilism.  Whole days I would be gone, starting
in the comparative cool of the morning and returning as the sun set over a flat
horizon an incomparable distance away.  After school, after work, on the
weekends I would always seem to wander off a bit farther, never wanting to turn
back.  I would walk with water always, big jugs that I would lug along which
would start with ice and slowly warm to tepid, and then the hot before I could
quaff enough to sate the heat which dripped down my legs and dried crisp and
salty on my neck.  Yet it didn't matter.  Nothing mattered out there except to
keep going, to stay in motion for its own sake.  There was nothing else to do.
Walking was meditation, a activity to involve nothing but that one action in the
legs.  It was a escape in slow motion.  In the desert one could think and feel
and contemplate, and it was at the same time the irrelevance of the quotidian in
the silence and stillness of the terrain.  

Casting out into this distances in some omphalocentric perambulation, I always
arrive back where I begun both in mind and body, back at my house, an old single
wide trailer with wall of faded and broken paneling and a roof with tires on the
top for a reason I never knew and never asked.  We had torn all the carpeting
out when we moved in and painted a giant ill-formed bird on the living room
floor, everything else blue because that was some paint we had. The front door,
glued two-by-sixes, faced a view miles in size, and I would stand on the
creaking gray wooden porch and contemplate the mountains distant, ever inviting
me to their feet for no purpose.

The first landmark that one would find after leaving down the dirt road which
stretched out from home was a dirt damn known as \textit{the pond}.  This
purported `pond' usually was devoid of water and instead just a dry brown basin
to stand as emblem of the constant aridity of the landscape.  When the rains
would finally come first hesitantly, then completely, myriad arroyos would fill
the pond with a fragrant brown liquid still foamy after its hurried course.  And
suddenly a life hidden beneath the mud would emerge in the form of a million
frogs whose frenzied calls could be heard through the walls at night.  It was
the sound of a life-cycle, the hurried croaking in the dark while the water
drained away till only tadpoles remained frozen in clay, fossils which would
fade in days and then be gone with no trace of the life dug beneath the baked
earth.  A vast system of scales would appear where once water had been, and dry
till each crack curled upward and detached like sun-burnt skin, a flaky sea
that crunched underfoot.

That pond has always stuck with me, this one symbol of the desert, of my
childhood taking the dogs out walking, our big lumbering dog named Spike and her
tiny Chihuahua companion running hard by her side.  Spike would disappear for
hours and finally return home panting dirty with some adventure that only she
would ever know, dirt and strange smells clinging to her sides.  Later I would
go dove hunting out at that same pond, hiding in the bushes on the side, or
ranging further in search of rabbits.

Beyond the pond lay The Big Arroyo as we called it.  This formed the meeting
point of many smaller channels streaming down from jagged peaks of the Organ
mountains to the east.  After the rain, it would fill and become a roaring river
for half a day taking rocks and bushes to some resting place I would never see.
I would ride my bike to the top of the hill where the road intersected this
nascent river, and steeling myself attempt to come crashing down the hill and
into the swirling currents, pushing the peddles as hard as I could to get to the
other side while grass and twigs strained themselves into my spokes and slowed
my tires.  If the current was to strong, I would find myself holding with one
hand to my bike as we both were taken down the stream and trying to struggle to
the side.

The Organ mountains were the orientation of everything, establishing direction
in a land flat.  I once climbed over these mountains starting from my front
door, half a day gone in reaching them, the rest to scramble up the cliffs and
down the other side.  I thought I could return in the day still, but ended
instead sleeping astride the mountain's saddle next to a smoky little fire while
I shivered at its side.  I finally came home to a search party come from all
around to save me, myself sunburned and dirty wandering up to see what all the
fuss was about.  I still remember my mother come running out of the front door
to hug me and make sure I was in one piece still.

\finish

\starbreak

I could often be found next door digging up my uncle's yard, uprooting mesquite
bushes, battling their spines, reworking the land to something less than wild.
My uncle and I had a struck a deal in which he would pay for my first computer
(a Gateway machine with a 10G hard-drive, a CD burner, a floppy drive, and a
dial-up modem) and I would landscape his yard at five dollars an hour over the
next year.  This was the point that I really fell into my life-long love of
computers.  I remember when I discovered the brilliant way in which a web page
worked, the systematic markup and the intricate way that everything came
together.  To think in the way that a computer thinks is to contort one's mind
into the ultimate systematic methodology of deconstructing problems.  These
problems are defined well enough on the computer, small microcosms with fixed
inputs and fixed outputs.  Yet, this type of thinking is something which will
scale upward also, structuring the ways that one thinks about more complex
aspects of nature.  I am always interested in the lack of overlap between
computer scientists and philosophers, as both seem to reinforce each other. To
my knowledge, the only philosopher of note has been Mencius Moldbug, computer
scientist and radical political theorist.  

\finish

Seven years I was home-schooled before I had taken the plunged and gone back to
public high-school starting freshman year.  It was a year of great angst and
exploration, this the first great Individuation from the outside.  Life prior
had been one familial, that of cousins and uncles primarily outside the nucleus
of mother and sister.  I would ramble between houses and desert, video games and
22 caliber rifles.  The heat was always present, a dry baking in the sun, in the
sand, a brief reprieve of the pool or a sprinkler till once again the dry heat
held me.  All this was my world till I turned fourteen and it opened to a whole
host of divergent lives which came and went in intersecting spirals.  The
cousins had all faded, all graduated or moved away, the diffusion of life in
full force.

I met Tony sitting on the floor in an indoor basketball court.  This was Phys
Ed, that awkward required class that all had to finally give themselves to, very
few willingly.  Tony, a strapping Irish boy just transferred sat on the floor
next to me. This was the part of the class that our body-building teacher failed
to learn the limits of the average high school student.  We did another set of
push-ups. Between sets, looking at my long hair and plaid attire standard at the
time, Tony's first words were, ``Hey\ldots hippy!''.  And thus began a strange
friendship.

Tony was less person than smirk.  He wold saddle up next to you and give that
laughing fist bump like you were in on a joke he was about to tell.  You were
being cultivated.  Tony's great joke was that he was always busy pulling
everyone inward, befriending all those in his orbit, charming each with the
comfort of his own skin.  Good looking, well built, he would work his bad-boy
shtick, the relaxed nihilism our age craves.

He would be no Good Boy.  Foster care at an early, he had floated between
houses, between moms, between other people's lives.  He had landed with a family
consummately middle class, a dog and no kids of their own.  Terry took them as
they where, well-intentioned and somewhat naive.  They would have him as that
nice middle class son, a set piece in their lives, a good son.  Yet real mom had
always told him, scolded him, yelled he was a bad boy.  Then came his new moms.
They kept packing him off to someone else, into someone other life, he never the
right boy, that good boy. 

And so he became the bad boy once more.  In the ferment of post-pubescent youth,
there is a draw to the rebel, a rejection of the order one has really never
known, a check he could now cash against his childhood.  To be an antinomian was
to have the ultimate individuation, to throw off even the society.  Everyone
wishes to be their own Napoleon and never realizes they are only a Raskolnikov
breaking something just to see what will happen. He was that fun force of chaos
that pulls inward like a black hole, everyone falling into Tony and his thrills.

\td{Rework all of this section}

Dropping out of school he took to check fraud after one day receiving a box of
checks in the mail addressed to a past tenant.  This was not his apartment, but
just that of a friend who let him stay. He would always find someone to let him
stay, inhabit there living room, eat their food and do their drugs and remake
their world into something like a nexus for all the traffic of Tony's life, he
the spider who had so craftily spun this web of people who came in pilgrimage
now.  He was the orchestrator, the center of this universe of people and
connections that flowed around him, bringing him all the things he needed or
wanted.  For years he rode couches and rode friends till the cops came and
kicked everyone out, leaving Tony always and somehow to escape and find a new
hole to squirrel himself into.  He was always fine --- till one day he wasn't.
Walking out of a gas station, a squad car awaited him.  Later he would learn
the man checking out before him was someone he had forged a check with weeks
before.  Tony had forgotten, but the man hadn't.  Tony spent my senior year in
jail.

\starbreak

Snider and I would sometimes skip school, escape the days expectations and
instead drive out into that still desert.  One of us always had pot stashed
among our things somewhere, a smelly sandwich bag hidden and a pipe in the glove
box. We would smoke and laugh and hackysack and hackysack for hours, circling
each other like pugilists, making trails in the dirt as this tiny ball the only
point of contact.  Finally hunger and pain would drive us from our trance and
back to my house and the world once more.  I would make a skillet overflowing
with stacked red enchiladas, refried pintos, top in lettuce and tomatoes and
sided with chips and toast with honey.  We would feast till we were both sated
and unmovable and contemplate our day and try to forget the morrow.

Other nights we would go scavenge wood from creosotes, build our little
stash into a fire and sit the night around with Old English 40s and laugh and
yell and scavenge more wood in the dark till we gave it all up and unrolled our
sleeping bags under a sky whose horizon was only where stars ceased to be, and
stayed the night in the open air and dreams of nothing.  The morning was always
gray in our haze with not a cloud in the sky.  The dust was in our hair and the
smell of smoke in our clothes, the night prior now memories and the future was
always crashing towards us still.

Then there was Jessica. Jessica and her tits.  She would pull them out at all
times, the young man's totem, huge and hanging like planets.  Pink pillowy
things with areolae bumpy and stretched, she would hike up he shirt and display
them for all occasions.  She would show strangers in cars next to ours, to
pedestrians, to anyone who looked interested.  We would all wait expectantly to
see when the time was right, when was the next showing. Then out they would
come, and all laughing we would secretly congratulate ourselves for something
not understandable.

One night she masturbated in my back seat while I drove down a road spotted in
street lights, Tony riding shotgun and smoking.  Light and dark would
alternatively show her face twisted in my rear-view mirror, moans from the dark
for a minute and then her eyes.  Tony and I were quiet, neither of us knew what
to do.  ``Are we almost there?'', she asked casually when she was done.

\starbreak

The years disappeared as the are wont to do, and senior year came rolling
around.  Tony was gone, Jessica moved away and only Snider and I still played
hackysack in the desert.  Things and friends shift strangely in ways subtle,
such that those who were close now seem far, and suddenly new faces appear to
frame your memories.  This year was different, I was going to college soon,
somehow I need to get my life in order.

\td{Rewrite this section}

High-school is the age which one can still be said to have ``promise'', some
mystical check to cash in the future.  My life rode two parallel paths, one of
promise, and one of dissolution.  I fought the last semester for an extra
Advanced Placement class, a fifth college-level chemistry instead of a standard
four class semester, or the two classes I needed to graduate. This is called,
``being an over-achiever.''  Yet, my days were often still spent in debauch, in
search of the good time of drugs and dissipation.  The last two years I would
come to class high most mornings having not done my homework.  I made no
apologies, took zeros on the work and got As on tests, and generally acted like
I didn't care a whole lot.  I had always harbored a secret comfort of being
certain I was the smartest student in most rooms I entered.  To ensure I never
had to test the issue, I held my grades lightly, uncaringly as a shield.

The economies of public schools are so very different from their purported
mission.  Those who are good at learning the material of which most classes
consist is something disincentivized by the majority of one's peers.  And this
is the paradox, the difference that one must always split between that of the
peer-group, and then that of the institution.  My game was to play both sides of
the field, to be both the rebel and the scholar.  To find approbation on both
sides and not have to choose.

Was it this that led to the great discovery of my youth, that I had to break all
molds if I was to not be found lacking in some metric.  I would break the
meter, reform the measurements, and by this I would avoid judgment, for to what
standard could I be held?

\finish

I got a job working in a Mail Boxes Etc my Sophomore year and worked the job
till I graduated.  The job was easy, sort some mail, run some postage, and run
the counter.  My aunt had introduced me to the owner, Ray, something of an old
hand at most things.  He was a machine gunner in Vietnam, champion weightlifter
in the army, and general cowboy.  He had worked for the oil company ran his own
businesses, and told stories sometimes about dating a stripper in Texas.  Now he
was running a little mailing store and dreaming about retirement.  He would roll
in with his new shiny car, perpetual cigarette hanging limply, and start playing
solitaire on the computer.  After surviving the jungles in Southeast Asia, he
didn't take anything too seriously these days, instead taking life as it came.

I would work after school most days of the week, a few hours every day before
the store closed up.  On weekends I would run the place by myself most days,
just me and the mail, building boxes and printing labels. Ray would pop in at
close, grab the money from the envelope hidden in the back after I counted
everything down twice, and we would head out for our weekend.

The store folded my senior year, the place not pulling in enough business after
the grocery store next door closed down.  Russ move on to a few more gigs before
retiring to a farm and a bunch of horses who now take up all his time.  The end
of a long life of adventure was the calm of a horse trotting beneath him, only
the desert spreading in all directions.

\finish

The three of them walked in to my chemistry class together.  Had they all met
before, or was it now that they where just drawn to each other? Nga and Nhi were
both from Vietnam, quiet girls demurring. Then Anja, from Thailand.

Anja had a charm probably known as spunk, some confidence I could never quite
find even if I tried to fake it. She knew only to act, not to think too much
about that action.  After me reciting the contours of a fear I had, she would
respond, ``So! Just do it'', and I know this no hyperbole.

There was something so attractive about her difference, about the
incomprehensibility of the foreign.  I would sometimes look at her askance and
think her another species, something superior, something pristine and yet
hardened to life in a way impossible to me.  She was this thing from the outside
which came here of her own accord from across the globe, from a life I would
never understand, from a place too different to comprehend.   

\finish

I had fallen for the first time, a puppy love maybe, something too hard to
analyze even with words for categories.  Snider and I were riding back one
night, and I, high and sleepy, sat shotgun concentrating my whole desire, my
whole being on trying to manifest Anja's presents right then, right there.  To
have her sit with me in that moment, and feel her warmth and her breath, to know
her somehow made solid with my touch, that I could make choate this emptiness by
the sheer force of desire.  Even today I remember that moment so distinctly, all
my will focused on the one thing least available to its force.

I closed my eyes to my dreams and the car hummed as we plied the way home,
Snider and I both quiet, lost to our fictions and dreams of a world not the one
we inhabited.  It was not long after she left again for Thailand, for her family
and her career.  Now she is a doctor, married to another doctor and both in
love, both the right fit.  Everything is right in her life, she happy and
content, and I am happy for her half a world away.

\starbreak

Prom night my Junior year, the triad of Tony, Snider and I decide it was time
instead for a boys night, a protest against our being single.  We would instead
take the night by storm and have a party our own.  There are certain dates which
are tyrannical in their implicit expectation.  As much as the idea is denounced
or the holiday derided, the lack of a feeling, an experience, is felt a loss
nonetheless.  It is to be sad for something one wants not to want.  Being alone
on December 25th will always feel a magnitude more empty than any other day; and
the same is true of those who try and deny Prom its due.

We had passed money to the right person and ended with alcohol, a thing
difficult to obtain normally.  After having scoured the town for parties, for
things interesting and coming up short, we decided to go visit Paul who lived in
the Porta-Potty at the park.

The three of us had met Paul a month prior.  After a good hack session going
late into the night, Tony took leave to relieve himself by visiting the line of
Porta-Potties set shoddily at the edge of the parking lot, the solution to a
broken bathrooms never repaired.  Tony, having approached the largest of the
lot, one made for the possibly disabled or the odd family, threw open the door
and started in only to be standing approximately on top of someone sleeping on
the floor.  After both had let out the appropriate expletive laden sounds, with
Tony having retreaded a good several yards in a motion not unlike teleportation,
everyone stopped to take stock of what had just happened.  The commotion had
brought myself and Snider over to have a look.  Tony gave us a recap of having
just kicked a man lying on the floor of the blue toilet, and after a strained
laugh, we conferenced as to what to do?  Was the guy drunk?  Was he dying?  Why
was he there?  ``Well, better ask'', Tony decided for the lot of us, and once
more approaching the potty knocked gently and bid the inhabitant a questioning
hello.  The response was normal enough for a man found sleeping in a park in an
extra-large portable toilet.  He even was nice enough to open the door after a
strained exchange of greetings and we came to know his name as Paul.

Paul was a man that looked like life hadn't given him a good day for a decade or
more.  Wearing dirty stain sweatpants and turtleneck fashionable for the 50s, he
told us of himself.  A man more in body than in mind, he seemed in a cycle of
the street.  His last attempt to find four walls not of plastic ended when his
last check was mugged, leaving him again to return to his Porta-Potty.  A simple
man, torn to pieces by life, left to the vicissitudes un-understandable to a
mind like his. Not only was it his mind, but also his body that had the worst of
it, with him missing a leg. Or such was what we assumed given his awkward gate
and the odd \textit{swoooshing} sound of a compressing hinge emanating from the
pants hiding his knee-like apparatus as he walked.  We bought him Taco Bell and
wished him well, for what else could we do?

Paul was the poor \textit{reprobatus} of some Calvinist deity, someone too low
in life to ever find a salvation.  These are the un-elect whom one feels helpless to
help, Paul floating in a world too far away to ever find his way back, in some unjust
judgment too hard for us to contest, too complicated for our own lives filled
already.  We drove away talking in low tones about how life can take a person to
such lows, unimaginable to us.  It was then that we decide on the one small
step, that one thing we could do for us as much as Paul, was to buy him a bungee
cord to keep his door closed at night.  The following day, we went on a mission
to the nearest home store to get this ersatz lock, gifting it to him with
several tacos and a twenty dollar bill.

It was to this Paul we now went with beer and pot and tacos in hand.  As it
turned out, Paul was nowhere to be found, and would never be found by us again.
We ate his Tacos and set up in the park anyway, ate and smoked and drank and
where happy in our youth and our debauch.

\chapter{Adobe}
\label{cha:adobe}

I came to Santa Fe following the whiff of truth, subject to the oft-mistake of
those who find philosophy in the heat of youth; to think life really reducible
to a predicate logic like Russel's \textit{Principia}. It was only to follow it
out far enough, and in the end would emerge some shinning Truth to calm that
unnamed disquiet which pervades life before growing fully into one's skin.  It
cannot be there are no answers, only experimentation, only questions fraught
with unclear answers.  The tyranny of our freedom has yet to find us.  The
disgust at our fall still convicts us.  And so the sensitive end with Plato
under our arm and a question on our face.  We come with hat in hand to ask the
ages why and what and to please explain.

And thus I came to St Johns College, a little liberal arts college tucked into
the hills of the mountain Atalaya, all adobe and curves and shrouded by Piñons.
Individuation comes in spurts, and the moment one moves from home is a maximal
point of change with little left in the way of the familiar.

Everyone took the same classes, as it was assumed that the history of western
thought progressed in the same manner regardless of where one came from or which
part one was interested it.  The whole Freshmen class would start with a class
on ancient Greek, the grammar housed in some indecipherable tome which one
learned to hate early.  Then there was math class, Euclid's Elements,
propositions devoid of number and only progressing with the logic inherent to
some ratio one had never considered.

It was the great homogenizing of Western thought before everything flew apart
into minority standpoints and the fracturing of reality.  This was the still
naive last point at which one could still speak of things like truth as being
more than subjective.
\finish

I was to have a dorm on the second floor, a tiny space almost a closet off the
main room of my roommate, a red-headed city boy from Seattle named Roy who was
always sure he's about to have a really good time.  Or at least half the time,
as Roy was two people, one studious and well spoken, the other an Id incarnate.
The first wore button up shirts and would speak in fully formed sentences, and
discourse on the art of life and love.  The second owned a Curious George body
suit and was out to cause havoc in any and all ways possible.  Curious Roy was
someone I would meet later in the year, the irrepressible alter-ego that first
peeks, and then comes crashing through a facade of maturity.  Yet both knew of
each other, and both approved.  Neither thought ill of the either.

After graduating from high school Roy had gone to India for several months
traveled around in trains feeling out of place entirely.  Coming back stateside,
he drove his 1954 Volvo out of the Seattle and east on a rode trip going
nowhere, circumnavigating the boarder of the country just to get a lay of the
land, sleeping in small town and parking lots and always pushing on to the next
place.  He would tell stories of nights spent alone in dirty motels watching TV
and drinking beer, feeling alone in the world, still out of place.  He came to
St Johns the next year with everything necessary to be a hipster before it was
invented.

I met him sitting in his dorm playing his guitar while I carried two
old CRT monitors through his room and set up a programming paradise in my
closet.  He looked on amused, both of us wondering what we would have to say to
each other, the mischief making adventurer from Seattle and myself, the desert
rat with too much computer gear and a love for the hackysack.  Turns out, we
both liked wine and books, and this was enough.  Despite everything, he was just
as confused about what he was doing there as I was, both of us looking confused
as we me it the hallway at 2:00 am trying to finish our Greek translations.

\finish
\starbreak

The weekends were the most interesting time, a set of parties always know about
and open to all somewhere on campus.  I never considered it at the time, but the
ability to discuss Aristotle while drunk at 3:00 in the morning with almost
anyone you encounter is a unique and cloistered experience.  

James, the lanky white boy with a mop of sandy hair was always on that damn
unicycle.  He was a communist, extolling Marx to anyone who would listen,
lecturing always on the \textit{prols} and class solidarity.  Class solidarity
on an elite liberal arts college would seem to be more bourgeoisie than we all
felt at the time, young kids still seemly subject to some power structure we
were figuring out how to be part of without our knowing it.  

He drove an old VW Bus perfectly restored and paint job pristine who everyone
was constantly offering to buy.  His response was a deep galloping laugh that
would erupt, sometimes at unexpected moments as though it even surprised him,
that giant Adam's apple bouncing up and down.  

Late enough on the Friday nights, inevitably the sounds of a desperate yelling
could be heard from some quarter of campus, him weaving around on the single
tire with a half-empty gallon jug to wine in one hand and the Communist
Manifesto in the other.  He belt aloud page after page while careening around
campus, never actually falling but forever a mere second away.

\starbreak

Rose lived downstairs with the girls, a strange sophomore who I would often meet
lying on a bench in the common area reading.  I would be out front hacking
furiously, finding the groove and really going for it, and there would be Rose,
lying on a bench reading, having found some \textit{ataraxia}.  She was quietly
irreverent, filled with these interesting quips I thought above me.  

I used to steal her pomegranates.  She would leave them on the windowsill to
ripen, a mere foot above the ground on her first floor dorm built have into the
hill.  I would pass by, taking a snaking way from class, in some mental reverie
and there would be her perfect pomegranates, lined up like soldiers, ripe
irresistible.  I would call her name, ``Rose!, Rose!'', and if no response, grab
the biggest one and abscond into the forest with it like some prize I had one.
It was only when the year was fading I fessed up, told her where her fruit had
gone all the while grinning.  We talked the rest of the day on my balcony, about
life, school, and all things, and she gave me a pomegranate willingly as parting
gift.  I never stole another.

Rose stayed at St Johns for six years, studying not only Greek but Sanskrit to
read the Eastern classics.  Now she is a Forest Ranger in Death Valley, riding
around in a space so big as to feel your finitude, the heavens never that high
or far.  I suppose that the end of all of philosophy if done with truth is to
escape to the desert and contemplate existence at its essence, at the point of
its origin, to find the \textit{axis mundi} at a place so sparse to exist is to
be monad. 

\starbreak

The beginning of break I had moved into a shed, living in the back of a house
across town, this small structure with no heating and only an extension cord run
through the door for power.  There were no windows, so if I wanted to see the
world outside, I would have to leave the door open to all the elements, good and
bad, cold and hot.  I put my computer on the floor and laid the sleeping bag
next to it and called it home.

I remember those cold morning sitting in the sun the door open, reading a book
on programming, dreaming about the future, of things not touched by my life
still small in the shed.  There is no universal metric for achievement, and this
is the bedeviling problem of life; one can construct things in an infinitude of
ways and none be called best.  There algorithm to sort the sole and find a
singular value, the answer.  

In the house lived a bunch of Deadheads, hippies who would live lives I would
never understand.  Some went to St Johns, some did not.  The Grateful Dead
played always, everyday, and at every chance.  The only break was the guy from
Mississippi who was played an amazing Banjo which would rule the house for a
minute before Jerry took back control.

\chapter{Forest}
\label{cha:forest}

To live in the forest is to know oneself not part of nature.  It is be sure the
universe has ends not for us.  Religion is waning in the West for a good
reason; when most of the world seems made to cater to our whims what is there to
remind us that we live only in a shell of our own creation.

I moved to the forest while patches of snow still stood brindling the
ground. Most of my possessions had been carted back down south with the help of
my mother, leaving only myself and an old forest-green army duffel bag which
dwarfed my frame as I struggled out of campus and into the forest to the North.

Santa Fe climbs only ever higher towards the peak of Atalaya, a 10,000 foot
mountain which pulls the whole city into a vertical, dripping down the side of
the Rockies. I would climb Atalaya in the hot summer days and sweat in the heat
till I reached the pines, their languid needles long enough to hide the sun.
The ground was red mostly, a deep hue that conjured the Southwest as some
archetypal place as much in mind as in local.

The forest begins first in Piñons, small shrubby trees with not even a trunk
standing straight, growing only in the elevations still part of the city.  Here
one still feels space, the trees too small to really enfold a person into the
landscape, the sky always too big. 

Then one moves upward and into ravines and out again, sandy soil and Piñon hulls
through veins of caliche, gray and yellow patches of hillside where the path has
been cut too deep. In the rain, these patches become slick as though oil were
poured on the path, slippery and impassable except by climbing rock to rock.
Then the trail out climbs the Piñons, giving way to Poderosas, big pines that
dominate the landscape at times, others withering to a forest of small stunted
trees born from seed scattered onto some hard hill open only to the elements
and set in rock and clay.  There were the places where one finds orientation, a
sudden break in the cloistering forest onto an escarpment for a moment before the
trail plunges back into a shaded way.

At first the trail would take inclines in iteration, one now and a reprieve
before the next, but soon the vertical was constant.  The climb was done bent
forward, my full backpack weighing heaver each step, my tilt ever more to
compensate.  Up the entwined ridges till they met in a final push to the top.
At this point, I would turn off the path and instead start down a hill each time
taking a different route to avoid any definition. I would weave downward jumping
rocks and brush till I hit the ravine and instead made this my guide, following
its course uphill.  The valley was narrow, a small concourse hidden between the
legs of a mountain a thousand feet still to its peak.  I trod this fold till the
valley opened just a hair, and there would be my hammock, army green and strung
between two trees invisible if not your objective.

I lived to leave no trace, so that I could pack and leave only a smooth bark
rubbed by rope where my hammock would sway in the breeze, myself tucked into two
sleeping bags inside one another, the hammock flaps drawn tight and tied
underneath to form a nightly cocoon.  This is to climb into the Gaia's womb,
soft and warm and yet the sound of the open silence so acute as to be untethered
from the temporal.

My hikes were half the time at night, the stars lightening the terrain just
enough to stumble forward by, a twilight world of a million other suns of some
alien calendar lost in a sea of night.  Light from eons ago, stars burned out
and collapsed into darkness and void from which even errant rays cannot stray.
On nights when the moon climbed full from behind Atalaya, it would come like day
on a different plane, the underside of earth mirrored and yet still twisted.  It
was the world in a hue alien and alone, myself always the only one plying the
silver ground to my hammock hung deep in the shadows of even this *sol
secundus*.

This small outpost in the Santa Fe National Forest I found and called my own was
but a bit stuff tucked in a set spot.  There is something liberating in the
lightness to which one can aspire, to a place which is no place, a non-home.  It
is to uproot an axis from which one has to revolve, and to instead float along
to where the moment pulls the universe around you.  It was as much to be at home
everywhere, all places equalized in their option. Now it is but for the
choosing.  

\finish

The question of the possible is so bound by episteme unseen the majority of the
time.  Cultures create systems so seamless that to find a gap is to have found
something brilliant, a pin of light leaking in from the unknown.  These are left
unusually only to those few who are pushed out not of their own will, but
because they where not quite ever in; these are the tramps and the vagabonds,
the addicts and the saints.  Then there are those few of choose, who wish to
experiment with life and cast aside all expectation and run headlong into some
unknown.  I would not call this finding the authentic as Sarte would, but
instead call it play.

The human is an amazingly adaptable beast.  The patterns of life necessary for
forest living are all in the end completely manageable.  One's days become
structured just as do all other lives.  There is an inversion which happens, the
hard things becoming easy and the easy becoming hard.  Water becomes something
precious, not had without a huge investment in time and effort.  Yet one has not
to worry about paying the water bill, or of the piping in the house, or of water
the plants in the yard.  Now the difficulty is in the schlepping requisite
bottles of the stuff to camp.  A calculus must always be made between the weight
and the want.  To carry more than necessary is perverse, but to not carry enough
is torture.  Most thing exist between these two points of the campers continuum,
and one is always and forever reevaluating the point one thought ideal even
yesterday.

There is a process of purging, something done not all at once but slowly.  The
notions of ``need'' and ``want'' must germinate, must find a level not had at
first.  One has to come to the want first through need, which is not natural in
the plenitude of modernity.  It is a conscious diminution, a curling into
oneself to see what is left.  Having seen this play out another decade, the mere
act has become the substance itself, a weird inversion that perverts the


\finish

Before I ever knew of Trader Joe's as a shopper or an employer, I knew of their
dumpster.  It was a shoddy thing out back of a strip mall with the ally facing a
baseball field.  The route I had made for myself circling town, hit first the
dumpster at the Joe's, then Whole Foods, then a bagel shop.  By the time
of my retrogress, I would be laden with food, baguettes ties all over my
backpack because the center was full, and I biking full bore back home.

The dumpster at Trader Joe's was the most taxing of my foraging, jumping into
the Dumpster and throwing trash bags around.  It was no precise operation,
bags of pure trash from the demo to something filled with food from someone too
lazy to spoil the one bad jar in a case.  At this time the donations system was
not up and running well, and I would find myriad foods, enough to live on if I
had no other food source.  I was once more a primate gatherer, sorting the bags
of trash in a dumpster and finding the means to live.

Then it was on to Whole Foods, who would compress most of their trash in a
compactor, an act which will always consider disrespectful.  They left their
day-old bread out back in carts though, tied and sitting next to the building.
I never went without a loaf, the most rich and filling thing left in the back of
a store because it was a day not sold and now free.  

One was only to ever increase oneself both in mind and body, not
necessary in corpulence, but in ones adjuncts.  Accouterments all expanded
oneself, a form of the extended phenotype which would push the self outward.


It was over this summer I also picked up a job at Trader Joe's for the first
time, a place whose gravitational pull I would find inescapable for the next
decade.  Growing up in southern New Mexico, the cult of Joe had not yet made it
that far south.  The company still was primarily based on the coasts, not yet
moving inland more than in a odd city here or other. Santa Fe was one of these
odd little outposts, no other stores existing in any nearby states excepting
Arizona.  It was not for another decade that Texas  would get it's first
store with Colorado following two years after that.

\finish

And so I stocked frozen food and learned the lingo of the bourgeois foodie. I
learned that a boule and hummus made a good snack but if you had a bit of


The summer made campus no longer mine, the dorms closed and filled with
interlopers from other parts of the US, all twisted histories I never understood
landing them here.  I would still roll up quiet in the night, take the track
from the runners field to the gym where I would lock my bike in front, the bike
rack always

\chapter{Hovel}
\label{cha:hovel}

The job was unclear but the pay was good.  I was to move mail up and down a
mountain, the mailman for the post-office.  I was to be a ``subcontractor'', which
at the age of twenty-one sounds as though one is doing something important, that
some trust has been given.  I gave up the life in Santa Fe,  packed everything
into the duff bag and rode south on the bus, the long track down the state.  Now
it was time for something completely new.

In two cities I would live, one in the mountains and one in the sand.  A truck
was to be mine, driven back and forth over eighty miles each day split in
halves, the day in Mayhill, the night in Alamogordo seven days a week.  My life
was to be bifurcated, one in a town so small that Main Street ended at 12, and
one in a town I kept at arms length.

Alamogordo, population 35,000, was a place no one thinks about living.  It is a
place to be born or a place to die, but both only unwillingly.  I would never
come to be a part of this town, always living on the edge of something here,
physically and mentally.  There was some disconnect to the place I cultivated,
in its way enchanting, unsure.  To be here was to be nowhere, to not belong to
the place I was.  I remember the de-centering, the feeling that I was both at
home and not any place in town.

The Sacramento mountains flank Alamogordo forming a backdrop to the west of
town.  The mountains rise first roundly, then starkly five thousand feet above
the city.  To the east, only the horizon hundreds of miles away forms the rim of
a flat desert without differentiation.  In 1945 the US government built a
sprawling military installation and missile range in vastness of the desert
outside the largest gypsum deposit on earth, White Sands Monument.  Not long
after, the first atomic bomb darkened the skies above this barren basin.  After
the war, V2 rockets were stolen and refined here.  What missiles now ply the
skies when the highway closes during test flights I leave to the imagination.

When the wind would kick up the sky to the west would turn white, gypsum clouds
thrown up from the dunes and hovering in the sky.  The dust was everywhere, a
powdery dirt that would coat everything.  Biking through this particulate wind I
would have to squint my eyes till nearly closed, they still caked with sand, red
and tearing by the time I arrived.  

I bought an old 10-speed Reighley racing frame while poking around a bike shop
and had it stripped and remade into a beautiful ride.  A Selle Anatomica saddle,
new cables, tires, spokes, chain, and grip tape.  After replacing the peddles
with toe clips I could hum around town with no special accouterments, still
sleek and fast. Tube-shifters makes the riding experience special, something
akin the driving a standard transmission.  The act of changing gears becomes
costly, it cannot be done with abandon like with brifters.  To pedal then is not
to run that one easy crank RPM too natural, but to feel the range, the hard and
easy, to start with strain and end puffing.

And puff I did, the city sprawling in a way that only America manages, roads
drifting off in all directions lined with shops and nothing in between, islands
of industry in a sea of sand.  I would bike Walmart on the northern edge of
town, and buy my dinner, some days soup, a baguette, or chips and salsa.  I
could eat for under three dollars most days, banking most of my money for the
future.  Eating food in the front of the store became a ritual, spooning some
cold food into my mouth from the jagged top of a tin can while watching all the
families living their lives, in and out and shopping and home.  I never desired
to be any of them, to live a live set-piece, something fixed and prescribed.  I
would rather be homeless, be free of something I was scared of, something I
could not name but would fight intensely to avoid.  Walking the middle class
neighborhoods at night, glancing in windows where every house had a TV blaring
and a whiff of stagnation, of the quiet life of desperation.  

Nights were spent on the south edge of town, locking my bike last light post
before I hit the darkness of desert and open land.  I crossed the street and
climbed a hill inaugurating the Sacramentos, higher and higher till I could see
dark city as a mirror to the stars above.  It was only in the night I could
climb and not be seen, the desert too naked to know any modesty about who beds
in her bosom.  Desert inclines are always steps, swathes of elevation which
suddenly level as though the hill were unsure of itself, before continuing again
its incline.  The first mesa was my choice, a struggle against bushes and cactus
till I topped the side and would check for cactus needles caught unbeknownst to
me somewhere.  In the summer I'd strip to my skivies, hang my sweat-laden
clothes in a bush, and unroll my tarp.  I made a game of toughening myself to
the ground, to the rocks and the debris beneath me, each night choosing ground
slightly less even, less inviting.  One night the rain had come cold,
unexpected, and sideways with the wind.  I rolled myself into a burrito, blue
tarp holding off the cold rain seeping into the old fabric, the old holes.
Lying on rocks in the desert, wrapped in an old found tarp while shivering
uncontrollably, all I could think was, ``Well, this will be a great memory!'',
and dissolve the present in the future.  Everyone has their own quantification
of ``tough'', and I chose mine abstrusely.

4:00 AM the alarm would sound, a battery-powered travel clock dinging away into
the open desert at the corner of the tarp.  Everything was rolled and strapped
to my backpack in the dark, then I would be off ambling down the hill to my
bike, always worried it wouldn't be there.  Once alight, the rest was easy,
flying down hill to my truck parked safely in town, to load up and start the
adventure up the hill again.

When the truck rolled up to the loading dock, there would already be waiting a
fleet of metal cages on wheels set to be rolled up the ramp and into the truck.
Everything was ratchet strapped to the walls and then off up the mountain, five
stops in all to dump a cage or two into the back of a glorified shed known a
town of population 100 as ``The Post Office''.  The postmaster wouldn't be by
till 8:00 AM when I would already be slipping into some dream again as the sun
was just really coming into it's own.

The truck emptied, my journey would leave me in Mayhill, a town overly gloried
by merely having a name.  It was more a slightly denser collection of houses
strung along the valley, most still trees and fields.  The only commercial
activity in town was a gas station, a feed store, and a barber.  On occasion a
restaurant would pop up for a few months in the empty building next to the feed
store before predictably going belly up soon after, a rotation of owner and
entrepreneurs who would not take no for an answer.  The place would be spruced
and cleaned, redone, and opened again.  The same handful of old cowboys, men
grudgingly retired but still wearing their spurs and wedded to a work they no
longer had, would amble back in like nothing had changed.  To venture inside was
to know one's self and outsider, not part of the morning huddle to talk over the
valley news, a horse dead, a buck seen, the way the world was on some tear
taking everything into cities and lives far away and never to be understood.
Show a man like this his house on Google Maps, a satellite view of his truck in
the driveway, and he will be on immediately about the second coming being near.

Past the quondam and future restaurant, a snaking dirt road led up an
undistinguished hill, rising up past the gas station and onto a flat bit of
grassy terrain, in the middle of which was something generously called a
``house''. What sat there rising above the overgrown grass was more a hovel, a
half-build shed over a twisted travel trailer.  The axles had been taken off the
trail and the whole thing set on the ground to rot away at a rate unequal to its
additions.  A gabled roof was set above expanding the footprint by half, with
walls coming out to enclose another room build onto the side.  The addition had
been framed and insulated, and then given up.  The insulation hung limply from
the unfinished walls, slowly shedding their pink innards to the air and onto any
clothes or a limb passed too close.

There were no shingles laid skyward on the roof, only graying plywood warped
and fraying at every edge.  The ridge-line was open to all elements, and When
the rain came, it would come in torrents down inside the walls, channel unseen
in everything, dripping out sockets and window frames.  A bevy of ice chests
\textit{sans} lids were ready at hand, hurriedly thrown in position as to
collect the emerging streams sprung inside the house.  Above the bed, water
would work its way between the paint and sheet rock, forcing down till it had
bulged the paint into a perfect breast hanging solitary from the ceiling,
lactating from its most prominent point down onto the bedding.  One ice chest
was for the bed, centered and left overnight in case the weather changed after
the mail and I migrated down the mountain.  The one time I left forgetting, the
bed was soaked through the next day and onto several expensive books I had left
lying on the sheets.  There was no safe place for books, each time the weather
came some new stream would appear in a place never wet before.  To this day many
of my water-warped volumes are from this time. 

The bathroom was a toilet and a old tub set right on the dirt.  The toilet was
the singular element hooked into a sewer system, the bath tub just piped out of
the hill a little ways down.  A wall had been framed up and put around the whole
affair, but these new walls didn't match up with the main gable, so with a
little corrugated metal and a few more 2x2s, something resembling a roof was
thrown up, never mind the large gaps.  The bathroom was well vented, especially
when there was a breeze.  For the floor, a refashioned piece of blue carpet
whose appearance was of second-hand at best, was cut to fit and laid down on a
smoothed dirt, still in places lumpy.  The wet ground beneath would wick into
the carpeting so that walking would make a squishing sound.  The day after a
fresh rain, I would arrive in the still breaking dawn, and there would be big
beautiful mushrooms sprung up overnight, arrayed around the base of the toilet
and growing right out of that blue carpet.

When the season had turned and winter came round I found a different set of
problems.  The bathroom was too porous to the outside to be heated, the walls
having no insulation and the gaps in the top distinguish it little from being
outside. The sole heat source was an old oil heater run off a butane tank
sitting outside. The thing would crepitate and hiss, but for all it's pretense
could only really change the temperature of that one small room while still
burning an insatiable amount of propane.  Now instead of collecting my morning
harvest of toilet mushrooms, I would flick matches till I got the pilot light
going on the heater, and then grab a trowel I had sequestered as a ersatz ice
pick and head into the bathroom.  The task at hand would be to chip enough ice
out of the toilet to allow it to function properly.  Sometimes this would just
mean stabbing at the ice in the bowl till a flush would not overflow after
filling the small remnant above the ice wall. Other times the job would require
several liters of hot water poured from the tea pot into all parts of the
toilet, and then flushing and hoping for the best.  I learned to control my
coffee consumption till I was sure everything was working properly, as my
plumbing was always fine after the second cup. 

In the winter I moved to the truck, the desert too cold for my gear.  It was
something like an unpainted U-Haul with a little door between the cabin and the
box on the back.  If the door was left open, I could sleep in the box with my
head not far from the door.  A second alarm was bought, this one set a ten
minute interval from the first.  Now when the first alarm would sound, I could
stretch an unwilling arm out into the cold enough crank the key set ready in the
ignition.  Retreating back to my sleeping bag, I would await the second alarm in
my half-dreams till the cabin was hot with the heater blowing full blast, the
engine now warm.  After a clamber into the cabin, always ungraceful at this
hour, the Thermos of tea made the evening before was perfect temperature.  Hot
tea in the cold of morning will always convince me that my life choices have not
all been bad.  That first drink still in the haze of dream. 
Pulling my bedding behind me into the passenger seat, I'd throw the truck in
drive and zoom off to get the mail one more time. 

Storms would cover the highway, a tenuous road in good weather and always
subject to rock slides.  The sheered faces were netted so boulders couldn't
launching themselves too far, yet this could never prevent their eventual
landing in the middle of the road.  It was no rare event to discover the road
nearly blocked by some few tons of stone just around a bend, now taking the
majority of both lanes.  The highway department would be out by the time I was
retracing my tracks, guiding traffic around and attempting to undo the damage.
Once up the hill, the road leveled off and turned into a snaking valley.  The
stream and cow pastures would entice all the wildlife in those dim hours while
the dew could still be seen hovering above the field full of deer.  The elk were
most normalized to the road, grazing just off its side, watching with
indifferent eyes as my truck flew past a few feet away.

\starbreak

In Mayhill, the last address on Main street belonged to Gruncle Brew, my
grandfather's cousin whose connection to the family become something fuzzy enough
for a neologism.  Gruncle Brew, the old cowboy, born in the dust bowl and pushed
west with the rest of his clan.  He landed in California like most did,
scratching his way as a young man till he was no longer.  I only know him in his
second act, retired welder, grader, cowboy, handyman, moonshiner, and chicken
fighter.  He had stories of wrestling cattle and welding up a moonshine stile in
Louisiana somewhere out in the bayou.  He had once build his own dump truck and
hired himself out moving dirt.  He was one of those people who was too poor to
go to school but too poor not to know how to do everything.

He still had his game cocks locked in pens in the back yard.  The hens
were worth more, breeding stock like money in the bank in Mexico.  Yet no one
came to buy a bird so far away and up a hill in the middle of
nowhere. Cockfighting was banned in every state now, had to go all the way down
to Mexico.  Brew would always be on about going down there, to go make some
money, but he never would.  He would raise his chicks and sell to no one, each
generation instead growing the number of cages.  In his old age he become
attached to his birds, too much time spent there with them, sitting out under
his elm tree.  Each morning he would prepare their food meticulously, passing by
each cage in turn, his little wiener dog named Princess always right behind. He
would ladling just the right amount for each bird with some formula he'll take
to the grave.  He couldn't fight them, yet couldn't give them up.

I learned to play music from Brew; or, rather, I learned to want to play music.
He had an old Martin with the frets worn down to the wood he would pull out and
suddenly the gruff voice would change and he was mellifluous.  All the
intonations of a hard life were gone, and instead he would croon some sweet
melody over that old guitar whose wood was old as he was, older.  His daddy had
bought his that guitar when they still had dirt floors in old Cali.

It was not the glories that he would remember though, it was only the faults.
Night were spent awake, thinking of the past, of all the moments gone wrong, of
all the things which should have been, been said and weren't, all the things
done now shameful.  It was the remnant of a life as most was even now the past,
his faith in heaven the only salvation.  He would endlessly talk of the bible,
of things yet to come, of times of perfection unable to be had in days such as
this, in the fallen world, a Kali Yuga of the age and himself.  The ceiling
above his bed was his tormentor, staring like the void back at him in the dark,
convicting, convicting.

He came to Jesus hat in hand, came for forgiveness of a life gone somehow all
wrong, the next life there to try again, to get in all right this time.  Somehow
things where always askew, how?, never sure.
\todo{complete this}

I would come wandering over and we would sit out back drinking beer, he filling
me in on the chickens, and I would tell him about what I was reading or learning
on the guitar.  It was about this time that the conversation would always
descend, spiraling into religion.  Bible references would flow off his tongue
like Monty Python quotes from my sister, everything a fit subject for them and
always thought apropos.  It is interesting to observe the mind of someone with a
reference so different, a starting point a world away like the sixth Euclidean
axiom somehow missed in class.

He had systems worked out, nights spent in the book of Revelation charting
genealogies and creating tables.  He had mapped the second coming, the Parousia.
He was sure the days of retribution were nigh, coming like a thief in the night
and it was for the living to watch for the signs and symbols, news he would
gather from his dissident cable box, piecing things together from Christian
TV programs discoursing on current events, to Russia Today and Al Jazeera.  He
had bought some cable box year ago that gave him access to all free Christian
and news stations, and he had made a patchwork of their content, half the time
arguing with them, there alone in his living room, the other half the time
importing some new story he had just learned.  We would go though his stations
sometimes in the morning, he telling me about the news the day before and its
implications to eternity and always so sure that this was the last days.

In the end, there was some asymptotic point we would approach from both
directions, I arguing one side and he the other. ``Look around! Obviously there
is a God!'', he would at some point finally declare after hours of circling the
subject, arguing church fathers and biblical sources.  I could only ever
respond, ``Look around! There is obviously nothing controlling all this!'' And
there argument would have to rest, for in the end, there is only the sense of
things, irrational and unmovable.

There are too many points in philosophy where one ends in a cul-de-sac, a point
at which there is no escape.  One can come to this from both directions, from
the experiential then to the logical, or the other way around.  Topics like free
will vs determinism and deity vs chance are such paths that end nowhere, alone
and irrefutable in a way that is no longer explanatory.  At these points, the
only thing left to do is retreat to something more general and start again, to
know there are a million rabbit trails that leads to nothing, to contradictions
inextricable like Godel show the Logical Positivists.  Philosophy is a blunt
instrument, and its applicability is limited to a scope smaller than it would
seem. 

The conversation having flagged, we would go to the kitchen to eat.  In his
refrigerator sat an eternal stew, a big pot with stains on the side, taken out
from its prominent place and placed on the stove to heat.  He would dip his bowl
each day, then rummage till he found some errant veggies or meat remaining in
the fridge, and chopping them imprecisely with his pocket knife throwing them in
the pot.  It was an evolutionary process, each day modified by the additions of
the day before, each day slowly becoming something new.  We ate that same stew
for the year I was there and I never ate the same stew twice.

\starbreak

In this job I learned to fear time.  The much to used adage that the mail must
always go through was true to my station. Three times late in two years and the
contract could go up for big again, and bye bye my job, and maybe the contract
owner, Jed.  The last driver had run through two of those leaving me none.  It
was for me to be there everyday, rain or shine, sleet of hail.  Jed have sold me
the truck, packed his things from the hovel, and moved to Iowa where he bought a
trailer and setup a new life.  I ran the mail and we both got paid, the Walmart
travel alarm the barrier between both our jobs and filing for unemployment.

My paranoia started with the naps.  After a 5:00 am shift, I would come rolling
into Mayhill and crawl into the bed housed in the hovel, not yet ready to
approach the day.  The sun would just be up over the mountains and I would be
once again in some dream of people past and time outside time.  I would crack a
lid, that tiny shaft of lift into my retina, and with a logic incoherent I would
come flying out of bed, sure I had missed that hateful alarm and the day had
come without me, I still supine and the mail waiting on the dock now for
someone else to come collect it.  Having hove myself into the middle of the
room, head not right but my pulse already up, I would stand dazed, and wonder
why I was already in Mayhill, what this meant and why I had stayed the night
here.  Then the morning would come crowding back like the end of a mushroom
trip, the contours of life reshaping themselves into some linear timeline. 


\finish


\starbreak

Days I didn't go in search of Brew, didn't find him sitting in his
camp chair out back by the tree, I would bike.  My tour bike was set at the
ready, always oiled and aired, cleaned and calibrated.  This place was a way to
the future, to leave this hovel to rot into the ground and cycle off in some
direction for no reason at all.  I wanted most to have no map, no timetable, and
place to call home.  I wanted to improvise each day, to choose who I was each
time I rolled into town for a day, an hour, or long enough just to peddle
through.  I wanted to exist only in the short-term memory of everyone I met, a
passing face to recall never again.  Wasn't this real freedom?  How could
anything else be, bound by everyone's idealization of yourself, a mental model
existing in someone's skull of you.

So I would push off from the hovel drive and onto the highways which spidered
the mountains and valleys of New Mexico, long winding roads with no one but me
there.  It would be hours sometimes before some care would crest the hill ahead,
a dot still, and grow till it finally passed, only to change octave and recede
till unseen around a bend.  Mostly the roads where mine, constructed for me to
find a lonely happiness out there peddling with all I had.

I would each day push to find that place of remove, the place after the
endorphins fade, after the burn of muscle dims, after everything had been given,
and there was nothing left but that crank turning, turning, turning.  It was the
meditation of action, the silencing of mind my having no energy to think of the
past or the future, it was to be empty.  It was to find nirvana, not that
heavenly bliss nonsense, but the snuffing out, the nothingness of the I.  

Then one day I pushed too hard, too fast, too far.  I climbed to Sunspot, an
observatory forty miles of mountains and valleys three thousand feet above me.
I rode so well that day, there and back in five and a half hours. It was also
that day I partially ruptured my quadriceps tendon in my left knee and killed
all the plans I had made for myself.


\chapter{Overlook}
\label{cha:overlook}


Santa Fe has some mystic call in the old architecture still stained with the
centuries, it betrays its age in the wrinkles of twisting roads, of a place
preserved and out of time.  In 1598 Juan de Oñate and his *Conquistadores* has
reach the limit of their greed and gave up on El Dorado, instead setting up an
outpost at Ohkay Owingeh where he would call the place Santa Fe de Nuevo
Mexico, a new province for Spain and for the glory of the Empire.  Later the
capital was moved to a new city at the feet of the Sangre de Cristo mountains
called *La Villa Real de la Santa Fe de San Francisco de Asís*.

The city still smells of Conquistadors and Acoma, of the fervent faith of
Catholic priests set in a desert land, sent by their god to contend with those
who lived pastoral lives in an age lost even then to time.  The Pueblos, brown
houses of mud build to house mystics all, closed in the earth except by a hole
pointing to the sky.

The pull to Santa Fe is that of gravity for the broken.  The earth here calls
across the continent to all those who must begin again, to remake life this
second time.  They come here not knowing why, but come they do and end here in
adobe walls waiting, the kiva still unlit, for some spark to turn the tide of
life.  For some it does, the spirits in the air right for some transfiguration.

I came again to Santa Fe to rebuild myself, to rebuild the body both in flesh
and mind.  I had come to consider myself first a broken thing, an incomplete
life that was filled first with problems, second with aspirations.  I had lost
the sense of self-actualizing, that pure will could indeed move the world if I
could just concentrate it in sharp enough a point.

My mom and I drove the barren back roads up to Santa Fe with a truck of camping
gear and a bag of clothes.  I had bought a plastic bin which was to house a
handfull of books in the uncertainties of the forest.  I had come back to try a
second time, to make my start this time further up the mountain, this time on a
ridge that brought the whole city into view, a set of blurry colors so
indiscreet as to give me proper perspective.  It was the view from Olympus, all
below washed into broad strokes of a landscape larger than could be seen from a
point of differentiation.  

The desert was too much, living in my tent at the edge of down, eating Ramen
noodles cooked over a camp stove swiveled onto a canister and set in the sand.
My days to too familiar, progressive for a tide, but then lost to the routine.
It was time to escape once more, to remake myself once again.

The cabin construction had set my bank account in order once more, I had means,
if only slight.  Santa Fe was known, a job was known (if not had), myself just
in need of time still to heal, to stretch the tendons in my knee to the point
which they would mend complete. I had steeled myself, and set a date, hoping
life would accommodate itself to decisions *fait accompli*.  And thus I had
packed my tent and left the desert abode without looking back.  It was an ugly
desert not to miss, too close to modernity to be alone and too far away to
sustain some goal in life again.

We had arrived in the light day, we driving the city on a tour of memories.  The
mind lets drift memories so well, they slide into something not itself but like
through a glass darkly.  The old curving streets where I had first zipped
through town by cycle were never quite the same road, life having remade the
past in bits, small additions and subtractions, till the *ding an sich* was had
receded into the years and hid beneath so many memories of a time life was good.

Things had changed some, the coffee shop closed, my bookshop now a furniture
store. I went only once to St Johns, and could not go back.  The place was too
weird, too filled with some sense of a perfect past that I would not disturb
with the present, with the new buildings, with the new students all unknown to
me.  They would forever be unknown to me, myself the perpetual outsider now,
myself the interloper come from a world outside and known so by all.  A knit
community is as much family as barrier depending on one's relation.  I left
campus with a strange sadness sitting somewhere inside me that would keep me
away forever after.

It was not till night that we would find the old trail, I carrying that same old
duffel bag last time I lived here, mom schlepping the gray tub filled half full
of books.  I had gone soft, this time wanting something more than just the
hammock.  I would have instead a place to sit inside, a collection of things I
enjoyed and something of a spot to make my own.  We set up in a valley as
before, but this time close the edge of the forest.  Arranging my new life in
some simple way, I made the best of my tiny spot half on a hill will everything
I owned always rolling away.  We called it good enough and my mom left the next
day having helped deposit me and affairs all in the forest half a state away,
and then she headed south again, back to some semblance of normal life.  

\chapter{Stairs}
\label{cha:stairs}

The old man stares blankly into the alley. Every afternoon he will be there,
metal gate open, he in the breach, dead eyes languid and incurious.  He stands
there, shirtless, belly sagging, his left foot propped on his other leg while he
looks around the tiny alley only a couple meters to the other side.  He looks
not as someone who cares to see, but instead just standing, idle, life now a
holding pattern waiting only for its culmination.

I walk past and he averts his eyes.  Three feet away now I find my gate and
fumble for my keys.  He stares past me, I always in his periphery while we both
stand there, I working angrily the lock on the gate.  We play a game of eye
contact, he skirting my gaze, I following his eyes trying to catch a glace.  The
gate finally open now, I complete my ingress with a laughing salute from behind the
bars, doff my cap and look directly, asking.  He nods ever so slightly as I turn
away, face still slack, giving finally into my persistence at the final
moment.  We will play again tomorrow. 

The boundary between inside and outside is less strict here, one flowing into
the other easily.  Doors are always open, air-conditioning rare and kept only in
the bedroom.  Drying clothes frame everything, bits of life left swaying for all
to see.  No one here is hurried, days coming on at some fixed coasting
speed. Yet everything is alive with humanity, millions of people all 
concourse with each other, each a node on some graph filled to almost solid.

There is a draw to the city that can only be felt in Asia, an emptying of the
countryside, a compacting of humanity into some tight mass.  In America, we
idealize rural life still in ways which will never let us understand the ever
growing slums of all developing countries.  Almost all those who live there
where not born in the complex systems of hovels and makeshift shelters, 
fractals of lives and ad hoc communities.  These people came here of choice, set
up their tin roofed plot and began again.  The slums never drain, for each who
makes it out, and most in time seem to, two new residents will replace them.
The slums are but a layover to a life of the city, of something less stultifying
than a shack in the country and some yearly cycle which will never change.

Saigon streets team with cars and motorbikes, the sidewalks crowded and packed.
There is seldom a place to stand out of the way while on street level, To be
outside is to be a transparent eyeball out of sheer necessity, the instinct for
self-preservation.

The first thing one has to learn to do in Vietnam is cross the street.  It
requires a confidence that does not come easily for the Occidental requiring a
faith he lost generations ago after the west was won and everyone settled down
to a middle-class life.  Here, all things moves in ways unquantifiable, a system
forever opaque in expectation and action.  Traffic is at once something one
could reasonably call a free-for-all, or in the same moment something so planned
as a minutely choreographed set of intertwined actions.  Maybe it is an emergent
system similar the human body, each cell doing its small bit until one day
sentience studies itself.  If I was to bet, higher consciousness is most likely
to exist in the network of minds on motorbikes all is some hurried mass to get
to a place somewhere else.

The city sits in a place between two worlds, one built by a generation which
knew only deprivations, poverty a constant and mobility unknown.  The city still
feels of this age, the small streets made for bicycles at best, most selling
small bits of food on their doorstep to those who wonder by or whatever they
managed to happen into.

Now modernity competes with this old world which is passing away into the
memories of a generation.  Old buildings get new shop fronts, everything painted
clean new colors while still under rusting roofs just out of sight.  Then one
day the whole building is gone and weeks later a new structure has come to take
its place among a city one can feel striving.  It is the continuous churn of
time condensed.

Yet city is slowly spacing itself out.  At the turn of the century, the density
was twice that of what it is now, a city five-fold the size.  Its landmass grows
twice as fast as the population, everyone now mobile with a machine between
their legs which can power them to a destination miles distant.

\finish

\starbreak

I don't feel as though I fit in here. I pace the room wondering about the years
gone too quickly, about how they all just disappeared into some haze of
indefinite moments, recollections of people and places unconnected to here and
now.  To leave is to break with people in more ways than distance.  To explain a
new life share at no points is almost impossible.

Who comes tramping across continents to set up in a country as bizarre as
Vietnam?  Why does this city in the swelter of equatorial heat draw in those
born in the bright cities of Europe?  It must be to be out of place.

I came here to break the complacency of a normal life, of an existence easy,
set-piece, understandable.  Here is to live in the ``Uncanny Vally'', no never
be at ease with the world around, to always be on edge as though one was in the
shadow of Kafka's castle, everything sliding along in a perfect pace but forever
sideways to a person outside the internal logic.

I live now in a top hat, a room five flights up built standing alone on a
structure vertical.  Houses here are as though the shotgun of New Orleans were
turned on its side and a staircase stuck on the side.  The neighborhood is all
strings of rooms stacked and a common set of stairs gluing the whole thing
together.

The house is filled with Europeans.  Five of us climb these same steps each
day, everyone living some variant schedule that keeps us all apart.  Mostly the
stairs are empty, the dark quiet reigns, only the sticky silence of bare feet
on the tiles as I trod down to the kitchen in a descending spiral.  The kitchen
is quiet as always, a chance meeting rare.  No one cooks or cleans, everything
abundant right outside the door.

The house has the silence of a meditation, no one daring to disturb the
stillness.  The heat keeps us hidden in our rooms, tucked away from each other
in the cool of the air conditioning. All us preserved in a stasis here,
separated from something vital.

Work is too easy for most to require real investment.  A few hours a day will
pay the bills, the remaining hours a bounty of excess.  The maid comes thrice a
week to clean and wash and put everything in our lives in order.  Food arrives
for most by motorbike, ordered from bed with a handy app.  All things are easy.
Days drift by in the quiet with nothing to cling to, we all passive in our
existence, in an atmosphere of slow dissolution.

In Australia the common aches and pains of everyone, sunburned and exhausted,
cooking some poor slop and the end of the day before the heaviness of the
setting sun drew us all early to bed.  Here in the heat, we have to strains to
bind us, all our lives instead autonomous and self-contained.  Here we come as
interlopers and remain always as such, a culture and a language barrier too
broad to dissolve.

Have I come here also in dissolution?  I once more rehash my years and wonder
what the summation describes.  Spuds and I sit quiet, I scribbling and she
reading, time now for us too some abstraction.  The shadows ingress until they
take the day complete, and we will still be sitting here, another revolution in
a cycle coming unbound and instead spooling off into the aether.  I sleep in
fits, questions rife with only the vagaries of dream as answer.

My scroll grows longer, not yet long enough to contain the world, barely enough
to contain myself in outline, a mere shadow on the wall of the cave cast in
these quixotic pages.  Yet I scribble on, waiting for some moment of inspiration
to transform these lines into a whole.

\starbreak

Like most other countries, the associated food which comes to define the cuisine
in foreign places is not the thing most eaten back at home. With Vietnam, its
two most known exports, Pho and the Banh Mi, are only a small tip of a menu so
vast as make these two items unimportant to the whole corpus.  Why is it that
counties who rely on the import of food from the world likewise only has room
for some small set of dishes?  Is it that there is no demand for the
complexities, or is it that each new restaurant only copies the one before it
until there is a self-enforce monopoly of only a select few dishes arbitrary in
origin?


\end{document}
