\documentclass[ebook, 11pt, openright, onecolumn]{memoir}

\usepackage[utf8]{inputenc}
\usepackage[english]{babel}
\usepackage[final]{microtype}
\usepackage{todonotes}

\setlrmarginsandblock{0.15\paperwidth}{*}{1} % Left and right margin
\setulmarginsandblock{0.2\paperwidth}{*}{1}  % Upper and lower margin
\checkandfixthelayout

\def\signed #1{{\leavevmode\unskip\nobreak\hfil\penalty50\hskip2em
  \hbox{}\nobreak\hfil(#1)%
  \parfillskip=0pt \finalhyphendemerits=0 \endgraf}}

\newsavebox\mybox
\newenvironment{aquote}[1]
  {\savebox\mybox{#1}\begin{quote}}
  {\signed{\usebox\mybox}\end{quote}}

\title{Moments}
\author{Tyler Thomas}




\let\cleardoublepage\clearpage

%%% BEGIN DOCUMENT
\begin{document}


\maketitle

\frontmatter

\null\vfill

\begin{flushleft}
  \textit{Moments}
  © COPYRIGHT Tyler Thomas
  
  
  ISBN--INFO
  
  ISBN--13: 
  \bigskip



  
  
  ALL RIGHTS RESERVED
  
  
\end{flushleft}

\clearpage
\tableofcontents
\clearpage

\mainmatter
\chapter{Desert}
\label{cha:desert}
\begin{aquote}{Latex Memoir Class Documentation}
  \textbf{memoir}, \textit{n.} a fiction designed to flatter the subject and to impress the reader.
\end{aquote}


I remember the desert most.  Days of walking down roads which led nowhere in the
hot sand and silence, my form of nihilism.  I would walk with water always, big
jugs that I would lug along which would start with ice and slowly warm to tepid,
and then the hot before I could quaff enough to sate the heat which dripped down
my legs and dried crisp and salty on my neck.  Yet it didn't matter.  Nothing
mattered out there except to keep going, to stay in motion for its own sake.
There was nothing else to do.  After school, after work, on the weekends I would
always seem to wander off a bit farther, never wanting to turn back.  Walking
was meditation, a activity to involve nothing but that one action in the legs.
It was a escape in slow motion.  Cast out into this distances in some
omphalocentric perambulation, I would in the end always arrive back where I
begun both in mind and body, back at my house, an old single wide trailer with
wall of faded and broken paneling and a roof with tires on the top for a reason
I never knew and never asked.  The front door faced a view miles in size, and I
would stand on the creaking gray wooden porch and contemplate the mountains
distant, ever inviting me to their feet for no purpose. 

I had not always wandered
\todo{Fill this out}

The first landmark that one would find after leaving down the dirt road which
stretched out from home was a dirt damn known as \textit{the pond}.  This
purported `pond' usually was devoid of water and instead just a dry brown basin
to stand as emblem of the constant aridity of the landscape.  When the rains
would finally come first hesitantly, then completely, myriad arroyos would fill
the pond with a fragrant brown liquid still foamy after its hurried course.  And
suddenly a life hidden beneath the mud would emerge in the form of a million
frogs whose frenzied calls could be heard through the walls at night.  It was
the sound of a life-cycle, the hurried croaking in the dark while the water
drained away till only tadpoles remained frozen in clay, fossils which would
fade in days and then be gone with no trace of the life dug beneath the baked
earth.  A vast system of scales would appear where once water had been, and dry
till each crack curled upward and detached like sun-burnt skin, a flaky forest
that crunched underfoot.

Seven years I was home-schooled before I had taken the plunged and gone back to
public high-school starting freshman year.  It was a year of great angst and
exploration, this the first great Individuation from the outside.  Life prior
had been one familial, that of cousins and uncles primarily outside the nucleus
of mother and sister.  I would ramble between houses and desert, video games and
22 caliber rifles.  The heat was always present, a dry baking in the sun, in the
sand, a brief reprieve of the pool or a sprinkler till once again the dry heat
held me.  All this was my world till I turned fourteen and opened my world to a
whole new host of lives which came and went in intersecting spirals.  



\chapter{Adobe}
\label{cha:adobe}
I came to Santa Fe following the whiff of truth.  I was subject to the mistake
of all those who find philosophy for the first time; to think life really
reducible to a predicate logic like Russel's \textit{Principia} such that if one
were to follow it out far enough, there in the end would be some shinning Truth
to calm that unnamed disquiet which pervades youth.  One cannot yet accept there
are no answers, only experimentation, only questions fraught with unclear
answers.  The tyranny of our freedom has yet to find us.

I ended in Santa Fe with too much stuff in tow.  Two CRT monitors, all my
programming books, and the general contents of my closet back home were to
remake my life in this new place.  The problem was to split the difference
between two emotions always operative at inflection points; that of finding some
bit of the familiar to grab onto while at the same time jumping into the
unknown.  Individuation comes in spurts, and the moment one moves from home
is a maximal point of change with little left in the way of the familiar.  Yet
youth loves adventure, 

One was only to ever increase oneself both in mind and body, not
necessary in corpulence, but in ones adjuncts.  Accouterments all expanded
oneself, a form of the extended phenotype which would push the self outward.

\chapter{Forest}
\label{cha:forest}

To live in the forest is to know oneself not part of nature.  It is be sure the
universe has ends not your own.  Religion is waning in the West for a good
reason; when most of the world seems made to cater to our whims what is there to
remind us that we live only in a shell of our own creation.

I moved to the forest while patches of snow still stood brindling the
ground. Most of my possessions had been carted back down south with the help of
my mother, leaving only myself and an old forest-green canvas duffel bag which
dwarfed my frame as I struggled out of campus and into the forest to the North.

The question of the possible is so bound by episteme unseen the majority of the
time.  Cultures create systems so seamless that to find a gap is to have found
something brilliant, a pin of light leaking in from the unknown.  These are left
unusually only to those few who are pushed out not of their own will, but
because they where not quite ever in; these are the tramps and the vagabonds,
the addicts and the saints.  Then there are those few of choose, who wish to
experiment with life and cast aside all expectation and run headlong into some
unknown.  I would not call this finding the authentic as Sarte would, but
instead call it play.

I had move into a cloistered life distant from the everyday already.

\chapter{Hovel}
\label{cha:hovel}
\todo{Chapter about mail run.}
\end{document}