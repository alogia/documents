\documentclass[ebook, 10pt, openright, onecolumn]{memoir}
\usepackage[T1]{fontenc}
\usepackage{graphicx}
\usepackage[utf8]{inputenc}
\usepackage[english]{babel}
\usepackage[final]{microtype}
\usepackage{todonotes}
\title{Moments}
\author{Tyler Thomas}

\setlrmarginsandblock{0.15\paperwidth}{*}{1} % Left and right margin
\setulmarginsandblock{0.2\paperwidth}{*}{1}  % Upper and lower margin
\checkandfixthelayout

\def\signed #1{{\leavevmode\unskip\nobreak\hfil\penalty50\hskip2em
  \hbox{}\nobreak\hfil(#1)%
  \parfillskip=0pt \finalhyphendemerits=0 \endgraf}}

\newsavebox\mybox
\newenvironment{aquote}[1]
  {\savebox\mybox{#1}\begin{quote}}
  {\signed{\usebox\mybox}\end{quote}}


\newlength{\drop}
%% specify the Webomints family
\newcommand*{\wb}[2]{\fontsize{#1}{#2}\usefont{U}{webo}{xl}{n}}
%% select a (FontSite) font by its font family ID
\newcommand*{\FSfont}[1]{\fontencoding{T1}\fontfamily{#1}\selectfont}
%% if you don’t have the FontSite fonts either \renewcommand*{\FSfont}[1]{}
%% or use your own choice of family.
%% select a (TeX Font) font by its font family ID
\newcommand*{\TXfont}[1]{\fontencoding{T1}\fontfamily{#1}\selectfont}
%% Generic publisher’s logo
\newcommand*{\plogo}{\fbox{$\mathcal{T}$}}

\newcommand*\td[1]{
  \todo[inline]{
     #1 
  }
}

%% Define some commands for marking up the chapters 
\newcommand*\starbreak{\fancybreak*{* * *\\}}
\newcommand*\finish{\td{ ----- Finish this section -----}}



\newcommand*{\titleAT}{\begingroup  % Anatomy of a Typeface
  \thispagestyle{empty}
  \FSfont{5bp} % FontSite Bergamo (Bembo)
  \drop=0.1\textheight
  \vspace*{\drop}
  \rule{\textwidth}{1pt}\par
  \vspace{2pt}\vspace{-\baselineskip}
  \rule{\textwidth}{0.4pt}\par
  \vspace{0.5\drop}
  \centering
  {\FSfont{5ml} % FontSite Mona Lisa
    \Huge MOMENTS}\\[0.5\baselineskip]
  {\FSfont{5ml}
    \Large a memoir}\par
  \vspace{0.25\drop}
  \rule{0.3\textwidth}{0.4pt}\par
  \vspace{\drop}
  {\Large \scshape Tyler Thomas}\par
  \vfill
  {\large\scshape $T^2$ Press}\par
  \vspace*{\drop}
  \endgroup
}


%% At some point kill the chapter numbering in this pagestyle
\addtopsmarks{headings}{}{
  \createmark{chapter}{both}{shownumber}{}{. \ }
}
\pagestyle{headings}

\let\cleardoublepage\clearpage

%%% ===================================================== %%%
%%%                     BEGIN DOCUMENT                    %%%
\begin{document}

\pagestyle{headings}
\titleAT
\clearpage

\frontmatter

\null\vfill

\begin{flushleft}
  \textit{Moments}
  © COPYRIGHT Tyler Thomas

  2019
  
  ISBN--INFO

  ISBN--13: 
  \bigskip

  ALL RIGHTS RESERVED
  
  
\end{flushleft}

\clearpage
\tableofcontents
\clearpage

\mainmatter

\chapter{Desert}
\label{cha:desert}
\begin{aquote}{Latex Memoir Class Documentation}
  \textbf{memoir}, \textit{n.} a fiction designed to flatter the subject and to
  impress the reader.
\end{aquote}


I remember the desert most.  Days of walking down roads which led nowhere in the
hot sand and silence, my form of nihilism.  Whole days I would be gone, starting
in the comparative cool of the morning and returning as the sun set over a flat
horizon an incomparable distance away.  After school, after work, on the
weekends I would always seem to wander off a bit farther, never wanting to turn
back.  I would walk with water always, big jugs that I would lug along which
would start with ice and slowly warm to tepid, and then the hot before I could
quaff enough to sate the heat which dripped down my legs and dried crisp and
salty on my neck.  Yet it didn't matter.  Nothing mattered out there except to
keep going, to stay in motion for its own sake.  There was nothing else to do.
Walking was meditation, a activity to involve nothing but that one action in the
legs.  It was a escape in slow motion.  In the desert one could think and feel
and contemplate, and it was at the same time the irrelevance of the quotidian in
the silence and stillness of the terrain.  

Casting out into this distances in some omphalocentric perambulation, I always
arrive back where I begun both in mind and body, back at my house, an old single
wide trailer with wall of faded and broken paneling and a roof with tires on the
top for a reason I never knew and never asked.  We had torn all the carpeting
out when we moved in and painted a giant ill-formed bird on the living room
floor, everything else blue because that was some paint we had. The front door,
glued two-by-sixes, faced a view miles in size, and I would stand on the
creaking gray wooden porch and contemplate the mountains distant, ever inviting
me to their feet for no purpose.

The first landmark that one would find after leaving down the dirt road which
stretched out from home was a dirt damn known as \textit{the pond}.  This
purported `pond' usually was devoid of water and instead just a dry brown basin
to stand as emblem of the constant aridity of the landscape.  When the rains
would finally come first hesitantly, then completely, myriad arroyos would fill
the pond with a fragrant brown liquid still foamy after its hurried course.  And
suddenly a life hidden beneath the mud would emerge in the form of a million
frogs whose frenzied calls could be heard through the walls at night.  It was
the sound of a life-cycle, the hurried croaking in the dark while the water
drained away till only tadpoles remained frozen in clay, fossils which would
fade in days and then be gone with no trace of the life dug beneath the baked
earth.  A vast system of scales would appear where once water had been, and dry
till each crack curled upward and detached like sun-burnt skin, a flaky sea
that crunched underfoot.

That pond has always stuck with me, this one symbol of the desert, of my
childhood taking the dogs out walking, our big lumbering dog named Spike and her
tiny Chihuahua companion running hard by her side.  Spike would disappear for
hours and finally return home panting dirty with some adventure that only she
would ever know, dirt and strange smells clinging to her sides.  Later I would
go dove hunting out at that same pond, hiding in the bushes on the side, or
ranging further in search of rabbits.

Beyond the pond lay The Big Arroyo as we called it.  This formed the meeting
point of many smaller channels streaming down from jagged peaks of the Organ
mountains to the east.  After the rain, it would fill and become a roaring river
for half a day taking rocks and bushes to some resting place I would never see.
I would ride my bike to the top of the hill where the road intersected this
nascent river, and steeling myself attempt to come crashing down the hill and
into the swirling currents, pushing the peddles as hard as I could to get to the
other side while grass and twigs strained themselves into my spokes and slowed
my tires.  If the current was to strong, I would find myself holding with one
hand to my bike as we both were taken down the stream and trying to struggle to
the side.

The Organ mountains were the orientation of everything, establishing direction
in a land flat.  I once climbed over these mountains starting from my front
door, half a day gone in reaching them, the rest to scramble up the cliffs and
down the other side.  I thought I could return in the day still, but ended
instead sleeping astride the mountain's saddle next to a smoky little fire while
I shivered at its side.  I finally came home to a search party come from all
around to save me, myself sunburned and dirty wandering up to see what all the
fuss was about.  I still remember my mother come running out of the front door
to hug me and make sure I was in one piece still.

\finish

\starbreak

I could often be found next door digging up my uncle's yard, uprooting
mesquite bushes, battling their spines, reworking the land to something less
than wild.  My uncle and I had a struck a deal in which he would pay for my
first computer (a Gateway machine with a 10G hard-drive, a CD burner, a floppy
drive, and a dial-up modem) and I would landscape his yard at five dollars an
hour over the next year.  This was the point that I really fell into my
life-long love of computers.  I remember when I discovered the brilliant way in
which a web page worked, the systematic markup and the intricate way that
everything came together.  To think in the way that a computer thinks is to
contort one's mind into the ultimate systematic methodology of deconstructing
problems.  These problems are defined well enough on the computer, small
microcosms with fixed inputs and fixed outputs.  Yet, this type of thinking is
something which will scale upward also, structuring the ways that one thinks
about more complex aspects of nature.  I am always interested in the lack of
overlap between computer scientists and philosophers, as both seem to reinforce
each other. To my knowledge, the only philosopher of note has been Mencius
Moldbug \todo{Write about Moldbug?}, computer scientist and radical political
theorist.

\finish

Seven years I was home-schooled before I had taken the plunged and gone back to
public high-school starting freshman year.  It was a year of great angst and
exploration, this the first great Individuation from the outside.  Life prior
had been one familial, that of cousins and uncles primarily outside the nucleus
of mother and sister.  I would ramble between houses and desert, video games and
22 caliber rifles.  The heat was always present, a dry baking in the sun, in the
sand, a brief reprieve of the pool or a sprinkler till once again the dry heat
held me.  All this was my world till I turned fourteen and it opened to a whole
host of divergent lives which came and went in intersecting spirals.  The
cousins had all faded, all graduated or moved away, the diffusion of life in
full force.

I met Tony sitting on the floor in an indoor basketball court.  This was Phys
Ed, that awkward required class that all had to finally give themselves to, very
few willingly.  Tony, a strapping Irish boy just transferred sat on the floor
next to me. This was the part of the class that our body-building teacher failed
to learn the limits of the average high school student.  We did another set of
push-ups. Between sets, looking at my long hair and plaid attire standard at the
time, Tony's first words were, ``Hey\ldots hippy!''.  And thus began a strange
friendship.

Tony was less person than phenomenon.  The consummate extrovert, he began
cultivating connections as soon as he arrived in a place, befriending all those
in his orbit, charming each with how comfortable he was in his own skin.  Good
looking and well built, he worked his bad-boy shtick constantly, that complete
lack of caring, the strange attractive nihilism of our age.  

His nihilism was earned.  Put into foster care at an early age, he had floated
between houses, between moms, between other people's lives.  He had landed with
a family consummately middle class with two dogs and no kids of their own.
Terry took them as they where, well-intentioned and somewhat naive.  They would
have him as that nice middle class son, a set piece in their lives.  Wasn't he
the bad-boy though?  Before foster care his mom mad always told him, scolded
him, made him feel he was that bad boy.  Now was he to be the good boy?  Was
this not to lie to himself, to play act in a world not his own?

And so he became the bad boy once more.  In the ferment of post-pubescent youth,
there is a draw to the rebel, a rejection of the order one has really never
known.  To be an antinomian was to have the ultimate individuation, to throw off
even the society.  Everyone wishes to be their own Napoleon and never realizes
they are only a Raskolnikov breaking something just to see what will happen. He
was that fun force of chaos that pulls inward like a black hole, everyone
falling into Tony and his thrills.

Dropping out of school he took to check fraud after one day receiving a box of
checks in the mail addressed to a past tenant.  This was not his apartment, but
just that of a friend who let him stay. He would always find someone to let him
stay, inhabit there living room, eat their food and do their drugs and remake
their world into something like a nexus for all the traffic of Tony's life, he
the spider who had so craftily spun this web of people who came in pilgrimage
now.  He was the orchestrator, the center of this universe of people and
connections that flowed around him, bringing him all the things he needed or
wanted.  For years he rode couches and rode friends till the cops came and
kicked everyone out, leaving Tony always and somehow to escape and find a new
hole to squirrel himself into.  He was always fine --- till one day he wasn't.
Walking out of a gas station, a squad car awaited him.  Later he would learn
the man checking out before him was someone he had forged a check with weeks
before.  Tony had forgotten, but the man hadn't.  Tony spent my senior year in
jail.

\starbreak

Snider and I would sometimes skip school, escape the days expectations and
instead drive out into that still desert.  One of us always had pot stashed
among our things somewhere, a tiny plastic bag hidden and a pipe in the glove
box. We would smoke and laugh and hackysack and hackysack for hours, circling
each other like pugilists, making trails in the dirt as this tiny ball the only
point of contact.  Finally hunger and pain would drive us from our trance and
back to my house and the world once more.  I would make a skillet overflowing
with stacked red enchiladas, refried pintos, top in lettuce and tomatoes and
sided with chips and toast with honey.  We would feast till we were both sated
and unmovable and contemplate our day and try to forget the morrow.

Other nights we would go scavenge wood from creosotes, build our little
stash into a fire and sit the night around with Old English 40s and laugh and
yell and scavenge more wood in the dark till we gave it all up and unrolled our
sleeping bags under a sky whose horizon was only where stars ceased to be, and
stayed the night in the open air and dreams of nothing.  The morning was always
gray in our haze with not a cloud in the sky.  The dust was in our hair and the
smell of smoke in our clothes, the night prior now memories and the future was
always crashing towards us still.


\finish
\starbreak


Prom night the triad of Tony, Snider and I decide it was time instead for a boys
night, a protest against our being single.  We would instead take the night by
storm and have a party our own.  There are certain dates which are tyrannical in
their implicit expectation.  As much as the idea is denounced or the holiday
derided, the lack of a feeling, an experience, is felt a loss nonetheless.  It
is to be sad for something one wants not to want.  Being alone on December 25th
will always feel a magnitude more empty than any other day; and the same is true
of those who try and deny Prom its due.

We had passed money to the right person and ended with alcohol, a thing
difficult to obtain normally.  After having scoured the town for parties, for
things interesting and coming up short, we decided to go visit Paul who lived in
the Porta-Potty at the park.

The three of us had met Paul a month prior.  After a good hack session going
late into the night, Tony took leave to relieve himself by visiting the line of
Porta-Potties set shoddily at the edge of the parking lot, the solution to a
broken bathrooms never repaired.  Tony, having approached the largest of the
lot, one made for the possibly disabled or the odd family, threw open the door
and started in only to be standing approximately on top of someone sleeping on
the floor.  After both had let out the appropriate expletive laden sounds, with
Tony having retreaded a good several yards in a motion not unlike teleportation,
everyone stopped to take stock of what had just happened.  The commotion had
brought myself and Snider over to have a look.  Tony gave us a recap of having
just kicked a man lying on the floor of the blue toilet, and after a strained
laugh, we conferenced as to what to do?  Was the guy drunk?  Was he dying?  Why
was he there?  ``Well, better ask'', Tony decided for the lot of us, and once
more approaching the potty knocked gently and bid the inhabitant a questioning
hello.  The response was normal enough for a man found sleeping in a park in an
extra-large portable toilet.  He even was nice enough to open the door after a
strained exchange of greetings and we came to know his name as Paul.

Paul was a man that looked like life hadn't given him a good day for a decade or
more.  Wearing dirty stain sweatpants and turtleneck fashionable for the 50s, he
told us of himself.  A man more in body than in mind, he seemed in a cycle of
the street.  His last attempt to find four walls not of plastic ended when his
last check was mugged, leaving him again to return to his Porta-Potty.  A simple
man, torn to pieces by life, left to the vicissitudes un-understandable to a
mind like his. Not only was it his mind, but also his body that had the worst of
it, with him missing a leg. Or such was what we assumed given his awkward gate
and the odd \textit{swoooshing} sound of a compressing hinge emanating from the
pants hiding his knee-like apparatus as he walked.  We bought him Taco Bell and
wished him well, for what else could we do?

Paul was the poor \textit{reprobatus} of some Calvinist deity, someone too low
in life to ever find a salvation.  These are the un-elect whom one feels helpless to
help, Paul floating in a world too far away to ever find his way back, in some unjust
judgment too hard for us to contest, too complicated for our own lives filled
already.  We drove away talking in low tones about how life can take a person to
such lows, unimaginable to us.  It was then that we decide on the one small
step, that one thing we could do for us as much as Paul, was to buy him a bungee
cord to keep his door closed at night.  The following day, we went on a mission
to the nearest home store to get this ersatz lock, gifting it to him with
several tacos and a twenty dollar bill.

It was to this Paul we now went with beer and pot and tacos in hand.  As it
turned out, Paul was nowhere to be found, and would never be found by us again.
We ate his Tacos and set up in the park anyway, ate and smoked and drank and
where happy in our youth and our debauch.

\starbreak

High-school is the age which one can still be said to have ``promise'', some
mystical check to cash in the future.  My life at this time rode always two
parallel paths, one of promise, and one of dissolution.  The last year of
high-school I fought for an extra Advanced Placement class, a fifth
college-level chemistry instead of a standard four class semester, or the two
classes I needed to graduate. This is called, ``being an over-achiever.''  Yet,
my days were spent mostly in debauch, in search of the good time of drugs
and dissipation.  I would come to class high most mornings having not done my
homework.  I made no apologies, took zeros on the work and got As on tests, and
generally acted like I didn't care a whole lot.  I had always harbored a secret
comfort of being certain I was the smartest student in most rooms I entered.
To ensure I never had to  test the issue, I held my grades lightly, uncaringly
as a shield.  

The economies of public schools are so very different from their purported
mission.  Those who are good at learning the material of which most classes
consist is something disincentivized by the majority of one's peers.  And this
is the paradox, the difference that one must always split between that of the
peer-group, and then that of the institution.  My game was to play both sides of
the field, to be both the rebel and the scholar.  To find approbation on both
sides and not have to choose.

Was it this that led to the great discovery of my youth, that I had to break all
molds if I was to not be found lacking in some metric.  I would break the
meter, reform the measurements, and by this I would avoid judgment, for to what
standard could I be held?  

\chapter{Adobe}
\label{cha:adobe}

I came to Santa Fe following the whiff of truth.  I was subject to the
oft-mistake of those who find philosophy in the heat of youth; to think life
really reducible to a predicate logic like Russel's \textit{Principia}. It was
only to follow it out far enough, and in the end would emerge some shinning
Truth to calm that unnamed disquiet which pervades life before growing fully
into one's skin.  It cannot be there are no answers, only experimentation, only
questions fraught with unclear answers.  The tyranny of our freedom has yet to
find us.  The disgust at our fall still convicts us.  And so the sensitive end
with Plato under our arm and a question on our face.  We come with hat in hand
to ask the ages why and what and to please explain.

And thus I came to St Johns College, a little liberal arts college tucked into
the hills of the mountain Atalaya, all adobe and curves and shrouded by Pinons.
Individuation comes in spurts, and the moment one moves from home is a maximal
point of change with little left in the way of the familiar.

I was to have a dorm on the second floor, a tiny space almost a closet off the
main room of my roommate, a red-headed city boy from Seattle named Roy who was
always sure he's about to have a really good time.  Or at least half the time,
as Roy was two people, one studious and well spoken, the other an Id incarnate.
The first wore button up shirts and would speak in fully formed sentences, and
discourse on the art of life and love.  The second owned a Curious George body
suit and was out to cause havoc in any and all ways possible.  Curious Roy was
someone I would meet later in the year, the irrepressible alter-ego that first
peeks, and then comes crashing through a facade of maturity.  Yet both knew of
each other, and both approved.  Neither thought ill of the either.

After graduating from high school Roy had gone to India for several months
traveling around in trains and feeling out of place entirely.  Coming back, he
drove his 1954 Volvo out of the Seattle and East on a rode trip to
circumnavigate the boarder of the country, sleeping in small town and parking
lots and always pushing on to the next place.  He would tell stories of nights
spent alone in dirty motels watching TV and drinking beer, feeling alone in the
world.  

\finish

\chapter{Forest}
\label{cha:forest}

To live in the forest is to know oneself not part of nature.  It is be sure the
universe has ends not for us.  Religion is waning in the West for a good
reason; when most of the world seems made to cater to our whims what is there to
remind us that we live only in a shell of our own creation.

I moved to the forest while patches of snow still stood brindling the
ground. Most of my possessions had been carted back down south with the help of
my mother, leaving only myself and an old forest-green canvas duffel bag which
dwarfed my frame as I struggled out of campus and into the forest to the North.

The question of the possible is so bound by episteme unseen the majority of the
time.  Cultures create systems so seamless that to find a gap is to have found
something brilliant, a pin of light leaking in from the unknown.  These are left
unusually only to those few who are pushed out not of their own will, but
because they where not quite ever in; these are the tramps and the vagabonds,
the addicts and the saints.  Then there are those few of choose, who wish to
experiment with life and cast aside all expectation and run headlong into some
unknown.  I would not call this finding the authentic as Sarte would, but
instead call it play.

The human is an amazingly adaptable beast.  The patterns of life necessary for
forest living are all in the end completely manageable.  One's days become
structured just as do all other lives.  There is an inversion which happens, the
hard things becoming easy and the easy becoming hard.  Water becomes something
precious, not had without a huge investment in time and effort.  Yet one has not
to worry about paying the water bill, or of the piping in the house, or of water
the plants in the yard.  Now the difficulty is in the schlepping requisite
bottles of the stuff to camp.  A calculus must always be made between the weight
and the want.  To carry more than necessary is perverse, but to not carry enough
is torture.  Most thing exist between these two points of the campers continuum,
and one is always and forever reevaluating the point one thought ideal even
yesterday.

\finish


One was only to ever increase oneself both in mind and body, not
necessary in corpulence, but in ones adjuncts.  Accouterments all expanded
oneself, a form of the extended phenotype which would push the self outward.


It was over this summer I also picked up a job at Trader Joe's for the first
time, a place whose gravitational pull I would find inescapable for the next
decade.  Growing up in southern New Mexico, the cult of Joe had not yet made it
that far south.  The company still was primarily based on the coasts, not yet
moving inland more than in a odd city here or other. Santa Fe was one of these
odd little outposts, no other stores existing in any nearby states excepting
Arizona.  It was not for another decade that Texas  would get it's first
store with Colorado following two years after that.

\finish

And so I stocked frozen food and learned the lingo of the bourgeois foodie. I
learned that a boule and hummus made a good snack but if you had a bit of

\chapter{Hovel}
\label{cha:hovel}

The job was unclear but the pay was good.  I was to move mail up and down a
mountain, the mailman for the post-office.  I was to be a ``subcontractor'', which
at the age of twenty-one sounds as though one is doing something important, that
some trust has been given.

In two cities I would live, one in the mountains and one in the sand.  A truck
was to be mine, driven back and forth over eighty miles each day split in
halves, the day in Mayhill, the night in Alamogordo.  Two shifts each day, 5:00
am to 7:00 am, and 3:00 pm to 5:00 pm, seven days a week except one shift on
Sunday.  My life was to be bifurcated between two lives, one in a town so small
that Main Street ended at 12, and one in a town I kept at arms length always.

I had bought an 10-speed old Reighley racing frame at a bike shop in town and
had it stripped and remade into a beautiful ride.  A Selle Anatomica saddle, new
cables, tires, spokes, chain, and grip tape.  After replacing the peddles with
toe clips I could hum around town with no special accouterments, still sleek and
fast. Tube-shifters makes the riding experience special, something akin the
driving a standard transmission.  The act of changing gears becomes costly, it
cannot be done with abandon like with brifters.  To pedal then is not to run
that one easy crank RPM too natural, but to feel the range, the hard and easy,
to start with strain and end puffing.

The Sacramento mountains flank Alamogordo, forming a backdrop to the west of
town.  These mountains rise first roundly, then starkly five thousand feet above
the city.  To the east, only the horizon hundreds of miles away forms the rim of
a flat desert without differentiation.  In 1945 the US government built the
largest military installation and missile range in the sprawling desert outside
the largest gypsum deposit on earth, White Sands Monument.  Not long after,
the first atomic bomb darkened the skies above barren basin. After the war
was won, V2 rockets were stolen and refined in this desert landscape.

\finish

And so for the day I was to live in this hovel, a half-build shed over a rotting
travel trailer.  The axles had been taken off and the whole thing set on the
ground to rot away at a rate unequal to its additions.  A gabled roof was set
above, framed and insulated, yet still leaking in all the elements.  There were
no shingles to face the sky, only graying plywood warped and fraying at the
edges.  When the rain came, a bevy of ice chests without lids were at hand to
collect the torrents of water that would come channeled down into house, into
walls and through crack and come trickling out in obscure corners.  Above the
bed, water would work its way between the paint and sheet rock, forcing down
till it had bulged the paint into a perfect breast shape, and from the nipple
drips would appear one after another.  I had to keep a bucket on the bed those
nights or the whole bed would be wet with lactation in the morning.

In this bed, I would never stay the night, always instead rambling down the hill
in my truck to return the mail.  Nights were spent on the edge of town, locking
my bike last light post before I hit the darkness of desert and open land.  I
crossed the street and climbed the hill opposite town.  In the desert, the
inclines are always steps, swathes of elevation which would suddenly level for a
minute as though it were unsure of itself, and then continue again its incline.
I would find the first mesa, the one had through a struggle against bushes and
cactus till I topped the side and would bed down with a blue tarp I had found.
In the summer I stripped to my skivies, hang my sweat-laden clothes in a bush,
and unroll my tarp.  I made a game of toughening myself to the ground, to the
rocks and the debris beneath me, each night choosing ground slightly less even,
less inviting.  One night the rain had come cold, unexpected, and sideways with
the wind.  I rolled myself into a burrito, blue tarp holding off the cold rain
seeping into the old fabric, the old holes.  Lying on rocks in the desert,
wrapped in an old found tarp while shivering uncontrollably, all I could think
was, ``Well, this will be a great memory!'', and dissolve the present in the
future.  Everyone has their own quantification of ``tough'', and I chose mine
abstrusely.

After a night hard won in the desert, I would rise at 4:00 AM, fold my tarp and
amble down the hill in the dark to my bike.  One alight, the rest was ease,
flying down the hill to my truck, load up and then go.  In the winter I would
sleep in the truck, set up my sleeping bag in the back, the cabin door open.
I bought two alarms, one to wake me enough to reach into the cabin and crank the
key, the second to wake me when the heater was blowing hot air.  I would crawl
out of my mummy bag and clamber into the cabin, pull my bedding behind me into
the passenger seat, and put the truck in drive.

After schlepping the mail up the hill, five stops in 80 miles, I would come to
rest in Mayhill, a town with two stores and a main street paved half way and
ending after address 12.  The last address belonged to Gruncle Bob, my
grandfathers cousin whose connection to the family become something fuzzy enough
for a neologism.  Gruncle Bob, the consummate cowboy, born in the dust bowl and
pushed west with the rest of his clan.  He landed in California like most did,
scratching his way as a young man till he was no longer.  I only know him in his
second act, retired welder, cowboy, handyman, brewer, and chicken fighter.  He
still had his game cocks, locked in pens in the back yard.  His hens were worth
more, breeding stock as money in the bank in Mexico.  Yet no one came, he still
loyal to an outlawed sport.  He would raise his chicks and sell to no one, each
generation instead a growing the number of cages.  In his old age he became
attached to his birds, the first thing a fighter can never do.  Each morning he
would prepare their food meticulously, passing by each cage in turn giving each
the right amount.  He could fight them no longer, yet could never give them up.

I learned to play music from old Gruncle Bob.  Or, rather, I learned to want to play
music.  He had an old Martin with the frets worn down to the wood he would pull
out and suddenly the gruff voice would change and he was mellifluous.  All the
intonations of a life lived hard were gone, and instead he would croon some
sweet melody over that old guitar whose wood was old as he was.

It was not the glories that he would remember though, it was only the faults.
Night were spent awake, thinking of the past, of all the moments gone wrong, of
all the things which should have been said and weren't, all the things done now
shameful.  It was the remnant of a life as most was even now the past, his faith
in heaven the only salvation.  He would endlessly talk of the bible, of things
yet to come, of times of perfection unable to be had in days such as this, in
the fallen world, a Kali Yuga of the age and himself. 

\finish

In this job I learned to fear time.  The tyranny of the clock I learned at this
age, as my job and my employers was on me.  To not get the mail was for the
contract to be up for bid again, the job gone from both me and the contract
owner.
\finish

\chapter{Stairs}
\label{cha:stairs}

The first thing one has to learn to do in Vietnam is cross the street.  It
requires a confidence that does not come easily for the Occidental just wondered
off the plane.  Everything moves in ways that one can never quite quantify, a
system still opaque in expectation and action.  Traffic is at once something one
could reasonably call a free-for-all, or in the same moment something so planned
out as a minutely choreographed set of intertwined actions.  In actuality is it
the definition of the emergent system. Like the human body, with each part doing
its small bit, something like a vast system emerges.

Like most other countries, the associated food which comes to define the cuisine
in foreign places is not the thing most eaten back at home. With Vietnam, its
two most known exports, Pho and the Banh Mi, are only a small tip of a menu so
vast as make these two items unimportant to the whole corpus.  Why is it that
counties who rely on the import of food from the world likewise only has room
for some small set of dishes?  Is it that there is no demand for the
complexities, or is it that each new restaurant only copies the one before it
until there is a self-enforce monopoly of only a select few dishes arbitrary in
origin?

\end{document}